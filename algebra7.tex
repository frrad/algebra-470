\documentclass[11pt]{article}



\usepackage[all]{xy}
\usepackage{fancyhdr}
\usepackage{amsthm}
\usepackage{amssymb}
\usepackage{setspace}
\usepackage{amsmath}
\pagestyle{fancyplain}

\begin{document}

\lhead{Frederick Robinson}
\rhead{Math 470: Algebra}



\title{Homework 7}
\author{Frederick Robinson}
\date{8 April 2011}
\maketitle




\section{Question 1}
\subsection{Question}
Let $A$ be an integral domain. Carefully work out the construction of the field of fractions Fr($A$) of $A$. You don't have to include all the details in your submitted answer, but please \emph{do} include the proof that addition and multiplication are well-defined, and also check at least one axiom involving each of addition and multiplication.
\subsection{Answer}
Let $(a,b), (c,d)$ be two representatives for a single element of Fr$(A)$, and $(e,f)$ some other element. Then, $(a,b)+(e,f) = (af+eb, bf), (c,d)+(e,f) = (cf+ed,df)$ since $ad=bc$. However, $(af+eb)df = afdf+ebdf = bfcf + bfed=bf(cf+ed)$. Similarly, $(a,b)(e,f)=(c,d)(e,f)$ since $aedf=bfce$.

We'll check that addition and multiplication are associative. Use the elements above but remove the assumption that the first two are representatives of the same element.  $((a,b)+(c,d))+(e,f) = (ad+bc,bd)+(e,f) = (adf+bcf+bde,bdf)=(a,b)+((c,d)+(e,f))$. Also, $((a,b)(c,d))(e,f)=(ace,bdf)=(a,b)((c,d)(e,f))$.


\section{Question 2}
\subsection{Question}
If $f: A \to K$ is a homomorphism from an integral domain $A$ to a field $K$, prove that $f$ extends to a homomorphism Fr($A) \to K$ if and only if $f$ is injective. Prove that such an extension is furthermore unique when it exists.
\subsection{Answer}
\begin{proof}
First, suppose that $f$ is injective.  Clearly an extension has $f(a/b) = f(a)\cdot f(b)^{-1}$ since homomorphisms respect inverses. This is well defined for our $f$ as $(a,b) = (c,d) \Rightarrow ad=bc$. Hence, $f(a)f(d)=f(b)f(c) \Rightarrow f(a)f(b)^{-1} = f(c)f(d)^{-1}$. All of the inverses involved are well defined by injectivity.

Conversely, if we have an extension to the fraction field, then it is injective, for if not there is a nonzero kernel, and the extension is not well defined.
\end{proof}

If the extension exists, it is unique, since any element of the field may be written in a the form $a/b$ and this maps to an element determined by the expression we used in the proof $f(a/b) = f(a)\cdot f(b)^{-1}$.

\section{Question 3}
\subsection{Question}
Find necessary and sufficient conditions on a homomorphism $f: A \to B$ of integral domains to extend to a homomorphism between their corresponding fraction fields.
\subsection{Answer}

$f$ extends if and only if it is injective, since $f$ is injective if and only if the composition $A \stackrel{f}{\to} B \hookrightarrow $Fr$(B)$ is, and by the previous exercise,  injectivity is necessary in sufficient for $F: A \to K$ to extend ($K$ a field).

\section*{Question 3*}
\subsection{Question}
If you know the language of adjoint functors, find a description of Fr as an adjoint functor to a forgetful functor. (This will include finding an appropriate choice of categories --- and remember that to specify a category you have to specify both the objects and the morphisms.)
\subsection{Answer}
The forgetful functor from the category of fields to the category of integral domains with injective morphisms has an adjoint which sends an integral domain to its field of fractions

\section{Question 4}
\subsection{Question}
Compute Spec $A$ (both as a set and as a topological space) for the ring $A = \mathbb{C}[x,y]/(x^2 + y^2 )$. Include a picture of Spec $A$ in your answer.
\subsection{Answer}
The relation is satisfied for any $x$, if and only if $y= i x$, and similarly for $y$. Therefore we consider the restriction to the ring $\mathbb{C}[x]$. In this ring the prime ideals are the 0 ideal, as well as the ideal generated by the elements of the form $(x-a)$ as shown in class. Topologically, this is just a copy of the complex plane, with a point corresponding to the root generating the idea, together with a point which represents the zero ideal.

Considering this in the spec of the entire ring, we just have the disjoint union of the two specs in for $\mathbb{C}[x], \mathbb{C}[y]$. The zero ideal  lifts to $(y)$ and $(x)$ respectively. Also, our ideal intersect at the point where the root is 0 for both since an ideal generated by $(x-a) \in \mathbb{C}[x] $ just lifts to $(x-a,y)$ and vice versa.
\section{Question 5}
\subsection{Question}
Is the following statement true or false: For a ring $A$, the space Spec $A$ consists of a single point if and only if $A$ is a field? If it is true, prove it. If it is false, give a counterexample, and then formulate and prove a suitably (and minimally) modified (correct!) statement.
\subsection{Answer}

The statement is false $\mathbb{Z}/4$ is a counterexample, as it has only a point in its Spec, but is not  a field.

If we modify the statement to read ``For a ring $A$ without nilpotents, the space Spec $A$ consists of a single point if and only if $A$ is a field" it becomes true.
\begin{proof}
For one direction, assume that $A$ is a field. Then, then the only prime ideal is the 0 ideal.

For the other, assume that there exists a nonzero prime ideal of $A$.  Then in particular we have nilpotent elements, a contradiction.	 
\end{proof}

\end{document}
