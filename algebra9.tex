\documentclass[11pt]{article}



\usepackage[all]{xy}
\usepackage{fancyhdr}
\usepackage{amsthm}
\usepackage{amssymb}
\usepackage{setspace}
\usepackage{amsmath}
\pagestyle{fancyplain}

\begin{document}

\lhead{Frederick Robinson}
\rhead{Math 470: Algebra}



\title{Homework 9}
\author{Frederick Robinson}
\date{29 April 2011}
\maketitle




\section{Question 1}
\subsection{Question}
Let $A$ be a ring. Prove that the following are equivalent:
\begin{enumerate}
\item \label{first} If $a \in A$ and $\mathfrak{p} \in $ Spec $A$ and $a \notin \mathfrak{p}$, then there exists an idempotent $e \in A$ such that $e \notin \mathfrak{p}$ and $a$ divides $e$.
\item \label{second} Each point of Spec $A$ has a neighborhood basis consisting of sets which are simultaneously open and closed.
\item \label{third} Given any two distinct points $\mathfrak{p}$ and $\mathfrak{q}$ of Spec $A$, we may find a decomposition Spec $A = X \coprod Y$ with each of $X$ and $Y$ being both open and closed, $\mathfrak p \in X, \mathfrak q \in Y$.
\item \label{fourth} Spec $A$ is Hausdorff.
\item  \label{fifth} Spec $A$ is a totally disconnected compact Hausdorff space.
\end{enumerate}
[Hint: First prove the equivalence of \ref{first} and \ref{second}, and the equivalence of \ref{third}, \ref{fourth} and \ref{fifth}. Then prove that \ref{second} implies \ref{fourth}. Finally show that \ref{fifth} implies \ref{second} using the general theory of totally disconnected compact Hausdorff spaces.]
\subsection{Answer}
$\ref{first}  \Rightarrow \ref{second}$
\begin{proof}
Let $\mathfrak p \in $ Spec $A$, and let $D(I)$ be a neighborhood of $\mathfrak{p}$. There exists $a \in I, a \notin \mathfrak{p}$, as  $\mathfrak{p} \in D(I)$. Now, fix some such $a \in I$. By assumption, there exists some idempotent $e$, $a | e$. I claim that Spec $(1-e)A$ is a clopen subset of $D(I)$. $(1-e)A$ is clopen, since $(1-e)$ is idempotent, proof in class.  Let $ q \in$ Spec $ (1-e)A$, then $e \notin q \Rightarrow a \notin q$. Thus, $I \nsubseteq q \Rightarrow q \in D(I)$ as desired. 
 \end{proof}
 $\ref{second}  \Rightarrow \ref{first}$
\begin{proof}
Choose $\mathfrak{p} \in $ Spec $A$, and $a \notin \mathfrak{p}$. Then, we have $\mathfrak{p} \in D(a)$. Now, by assumption, there exists a closed and open set containing $\mathfrak{p}$ and contained in $D(a)$. This set is of the form $(1-e)A$ for some idempotent $e$ by a proof in class. Since any point not in $(1-e) A $ is in $eA$, we have $a \in q  \Rightarrow q \subseteq eA \Rightarrow e \in q$. Thus, $e \in (a)$  and $a | e $ as desired.
\end{proof}
$\ref{third} \Leftrightarrow \ref{fourth}$
\begin{proof}
Note that \ref{third} is the definition of \ref{fourth}.
\end{proof}

$\ref{fifth} \Rightarrow \ref{fourth}$
\begin{proof} \ref{fourth} is just a weaker version of \ref{fifth} \end{proof}

$\ref{fourth} \Rightarrow \ref{fifth}$
\begin{proof}
Specs are quasicompact, so if Spec $A$ is Hausdorff, then it is a compact Hausdorff space. Furthermore, quasicompact in a Hausdorff space $\Rightarrow$ closed, and each $D(a) \cong $ Spec $A$. Thus every open set is also closed, and Spec $A$ is totally disconnected. 
\end{proof}

$\ref{second} \Rightarrow \ref{fourth}$
\begin{proof}
Spec $A$ is always $T_1$, thus, we can find a neighborhood of $\mathfrak {p}$ not containing $\mathfrak {q}$. Then, by our assumption, there is a clopen neighborhood of $\mathfrak {p}$ not containing $\mathfrak{q}$. So, Spec $A$ is Hausdorff.
\end{proof}


\section{Question 2}
\subsection{Question}
Let $k$ be a finite field, and let $A = \prod _{n=1}^\infty k$. Prove that $A$ is zero-dimensional, and that for each $\mathfrak{p} \in $ Spec $A$, there is an isomorphism $k \cong \kappa(\mathfrak{p})$.
\subsection{Answer}

Prime ideals of $A$ are of the form $(k,k,\dots,k,0,k,\dots)$. If  an ideal contains any nonzero elements of a component, then it must contain all of them, as the components are fields. Furthermore, if a prime ideal contains only the zero elements o more than one $k$ component, say the first and second, we consider the product of $(0,1,1,\dots)$ and $(1,0,1,\dots)$ to show that this is not prime.

We have that $A$ is zero dimensional, since for $a= (x,0,\dots) \notin (0,k,\dots) $, the element $e=(1,0,\dots)$ is idempotent, and $a$ divides $e$, since it's in a field. 

Finally, the isomorphism from $\kappa(\mathfrak{p})  = A/ \mathfrak{p} $ to $k$ is given by sending an element of $A$ to its value in on one fixed place in the product corresponding to the choice of  $\mathfrak p$.

\section{Question 3}
\subsection{Question}
Let $X$ be a topological space, and write $A  = \mathcal{C}(X,\mathbb{F}_2)$.
\begin{enumerate}
\item If $x \in X$, write $\mathfrak{p}_x := \{ f \in A \mid f(x) = 0\}$. Prove that $x \mapsto \mathfrak{p}_x$ is a continuous map $X \to $ Spec $A$.
\item Prove that $X \mapsto A$ and $A \mapsto$ Spec $A$ are adjoint functors between topological spaces and Boolean rings (i.e. rings in which every element is idempotent.)
\item Prove that the continuous map $X \to $ Spec $A$ is a homeomorphism if and only if $X$ is compact, Hausdorff, and totally disconnected.
\end{enumerate}
\subsection{Answer}
\begin{enumerate}
\item Consider an open subset $D(I) = \{ \mathfrak{p}_x \mid I \nsubseteq \mathfrak{p} \} \in $ Spec $A$. Then, a function from $I$ is not in each of the  $\mathfrak{p}_x \in D(I)$. That is, a function of $I$ takes value 1 at every $x$. So, $D(I)$ is the set of of $p_x$ for $x \in B = \{ x \mid f(x) =1, \mbox{ some }  f \in I\}$. Thus, the inverse image of $D(I)$ is just $B$, which is open, as $f^{-1}(1)$ is open for any fixed $f$, and $B$ is just the union of this quantity over each $f \in I$.

\item 
\item We saw in first part that the map is continuous. Thus, it is a homeomorphism if and only if it has continuous inverse. Fix some open $B \in X$, and consider its inverse image under the inverse map, that is $\{\mathfrak{p}_x \mid x \in B \}$. The inverse is continuous, if and only if this set corresponds to some open set in Spec $A$ for every choice of $B$. That is, if and only if $D(I) = \{ \mathfrak{p}_x \mid \mathfrak p_x \nsubseteq I \} = \{\mathfrak{p}_x \mid x \in B \}$ for any choice of $B$ open in $X$.

If some $\mathfrak p _ x \in D(I)$, there is some $f \in I$ with $f(x) = 1$. Furthermore, if some $\mathfrak p _ x \notin D(I)$, then  each $f \in I$ has $f(x) = 0$. Therefore, $B = D(I) $ if and only if $I$ contains a function which takes the value 1 on all of $B$, and 0 outside of it. As all functions are assumed to be continuous, this is equivalent to demanding that all sets be both open and closed.

Employing the first exercise, finishes the proof.
\end{enumerate}



\end{document}
