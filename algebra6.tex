\documentclass[11pt]{article}



\usepackage[all]{xy}
\usepackage{fancyhdr}
\usepackage{amsthm}
\usepackage{amssymb}
\usepackage{setspace}
\usepackage{amsmath}
\pagestyle{fancyplain}

\begin{document}

\lhead{Frederick Robinson}
\rhead{Math 470: Algebra}



\title{Homework 6}
\author{Frederick Robinson}
\date{14 March 2011}
\maketitle




\section{Question 1}
\subsection{Question}
Prove that induction is transitive.
\subsection{Answer}
For $H \leq K \leq G$ we want $Ind_K^G ( Ind_H^K V) = Ind_H^G V$.

\begin{proof}
This is equivalent to showing that
\[FG\otimes_{FK} (FK \otimes_{FH} V ) \cong FG \otimes_{FH } V.\]
But this is clear. 
\begin{align*}
FG \otimes_{FK } (FK \otimes_{FH } V) &\cong (FG \otimes_{FK} FK ) \otimes_{FH} V \\
&\cong FG \otimes_{FH} V.
\end{align*}
\end{proof}

\section{Question 2}
\subsection{Question}
Use your earlier computation of the character table of $S_5$ to compute the character table of $A_5$.
\subsection{Answer}
Recall that the  character table of $S_5$ is
\[
\begin{array}{|c|c|c|c|c|c|c|c|}
\hline
&1&10&20&30&24&15&20\\
&e&(12)&(123)&(1234)&(12345)&(12)(34)&(12)(345)\\
\hline
\chi_1&1&1&1&1&1&1&1\\
\hline
\chi_2&1&-1&1&-1&1&1&-1\\
\hline
\chi_3&4&2&1&0&-1&0&-1\\
\hline
\chi_4&4&-2&1&0&-1&0&1\\
\hline
\chi_5&5&-1&-1&1&0&1&-1\\
\hline
\chi_6&5&1&-1&-1&0&1&1\\
\hline
\chi_7&6&0&0&0&1&-2&0\\
\hline
\end{array}\]

First we need to get the conjugacy classes. The identity is still in its own conjugacy class. We use the orbit-stabilizer theorem as in class and observe that the conjugacy classes in $S_5$ made up of even permutations correspond to a single conjugacy class if they commute with some odd permutation. If they do not, then they break up into two conjugacy classes of the same size. 

The product of two disjoint 2-cycles commutes with a transposition, as does a 3-cycle. These are therefore still a single conjugacy class. The conjugacy class of  5-cycles however splits, as the centralizer of a given 5-cycle is  just the cyclic subgroup generated by this cycle. However, in $A_5$ this centralizer contains a different number of elements. 


Now, we note that there are 4 and 5 dimensional representations by the restriction argument from class. Then, since 
\begin{align*}
60 &= 1 + 16+25  + \sum n_i^2\\
18 &= \sum n_i^2
\end{align*}
the remaining two must have dimension 3.
\[
\begin{array}{|c|c|c|c|c|c|}
\hline
&1&20&12&12&15\\
&e&(123)&(12345)&(12354)&(12)(34)\\
\hline
\chi_1&1&1&1&1&1\\
\hline
\chi_2&3&a&b&c&d\\
\hline
\chi_3&4&1&-1&-1&0\\
\hline
\chi_4&3&e&f&g&h\\
\hline
\chi_5&5&-1&0&0&1\\
\hline
\end{array}\]

Now we use orthogonality of columns. First, use it wit the column corresponding to $(123)$. We get $|A_5:\left<(132)\right>| = 1^2 + 1^2 +(-1)^2+a^2 + e^2$,  so $a=e=0$. Now, using it with the column corresponding to $(12)(34)$ to get $|A_5:\left<(12)(34)\right>| = 4 = 1^2 + (-1)^2 + d^2 +h^2$, we see that $d, h$ are each one of $\pm1$. Since the sum $\chi_2 + \chi_4$ is $\chi_7$ from the character table of $S_5$ we must have $d=h=-1$.
\[
\begin{array}{|c|c|c|c|c|c|}
\hline
&1&20&12&12&15\\
&e&(123)&(12345)&(12354)&(12)(34)\\
\hline
\chi_1&1&1&1&1&1\\
\hline
\chi_2&3&0&b&c&-1\\
\hline
\chi_3&4&1&-1&-1&0\\
\hline
\chi_4&3&0&f&g&-1\\
\hline
\chi_5&5&-1&0&0&1\\
\hline
\end{array}\]

Finally, we may apply row orthogonality to $\chi_1, \chi_4$ to obtain
\[3+12 f +12 g -15 = 0 \Rightarrow f+ g = 1.\]
Now, since $\left< \chi_4, \chi_4\right> =1$ we have 
\[ 60 = 9 +12 (1-g)^2 + 12 g^2 -15 \Rightarrow g =\frac{1}{2} \left(1 \pm \sqrt{5}\right) .\]
Using the fact that the sum $\chi_2 + \chi_4$ is $\chi_7$ from the character table of $S_5$ again we now have
\[
\begin{array}{|c|c|c|c|c|c|}
\hline
&1&20&12&12&15\\
&e&(123)&(12345)&(12354)&(12)(34)\\
\hline
\chi_1&1&1&1&1&1\\
\hline
\chi_2&3&0&\frac{1}{2} \left(1 \pm \sqrt{5}\right)&\frac{1}{2} \left(1 \mp \sqrt{5}\right)&-1\\
\hline
\chi_3&4&1&-1&-1&0\\
\hline
\chi_4&3&0&\frac{1}{2} \left(1 \mp \sqrt{5}\right)&\frac{1}{2} \left(1 \pm \sqrt{5}\right)&-1\\
\hline
\chi_5&5&-1&0&0&1\\
\hline
\end{array}\]

Now we can just make an arbitrary choice of sign, since making the other choice just amounts to a relabeling.
\[
\begin{array}{|c|c|c|c|c|c|}
\hline
&1&20&12&12&15\\
&e&(123)&(12345)&(12354)&(12)(34)\\
\hline
\chi_1&1&1&1&1&1\\
\hline
\chi_2&3&0& \frac{1+\sqrt5}{2}&\frac{1-\sqrt5}{2}&-1\\
\hline
\chi_3&4&1&-1&-1&0\\
\hline
\chi_4&3&0&\frac{1-\sqrt5}{2}&\frac{1+\sqrt5}{2}&-1\\
\hline
\chi_5&5&-1&0&0&1\\
\hline
\end{array}\]


\section{Question 3}
\subsection{Question}
Let $H$ be a subgroup of $G$ and let $V$ be a representation of $G$. Prove that Ind$_H^G($Res$_HV)  \cong V \otimes$Ind$_H^G1$, where 1 is the trivial 1-dimensional representation.
\subsection{Answer}
\begin{proof}
We denote by $W = Fw$ the $FH$ module affording the trivial representation on $H$.
Manipulating the statement we want to prove a bit, we have.
\begin{align*}
Ind_H^G(Res_H V) &= FG \otimes_{FH} ( Res_H V),
\end{align*}
and
\begin{align*}
V \otimes_F Ind_H^G 1 &= V \otimes_F(FG \otimes_{FH} W)\\
&\cong (FG \otimes_{FH} W) \otimes_F V\\
&\cong FG \otimes_{FH} (W \otimes_F V) .
\end{align*}
So, it suffices to show that
\[W\otimes_F V \cong Res_H V.\]

I claim that $\varphi: W \otimes_F V \to Res_H V$ defined by $a w \otimes v \mapsto av$ for $a \in F$  is an isomorphism of $FH$ modules. First, we verify that it is a homomorphism
\begin{align*}
\varphi(g \cdot (aw \otimes v)) &= \varphi(haw \otimes hv)\\
&= \varphi(a w \otimes hv)\\
&= a(  hv)\\
&= h \cdot av\\
&= h \varphi (aw \otimes v)
\end{align*}

Also, it is surjective since the kernel is 0. Let $\sum a_i w \otimes v_i$ be in the kernel. Then, 
\begin{align*}
\sum a_i w \otimes v_i &= \sum w a_i \otimes v_i\\
&= \sum w \otimes a_i v_i\\
&= w \otimes \sum a_i v_i\
\end{align*}
and as $\sum a_i w \otimes v_i$ is in the kernel, we clearly have $\varphi(\sum a_i w \otimes v_i) = 0 = \sum a_i v_i $. Thus, our original element was $w \otimes 0 = 0$, and the map is injective. Clearly $\varphi$ is surjective, so it is an isomorphism of $FH$ modules, as claimed.
\end{proof}

\section{Question 4}
\subsection{Question}
Prove that Ind$^G_H 1$ is isomorphic to the permutation representation on the set $G/H$.
\subsection{Answer}
\begin{proof}
Denote by $V = Fv $ denote the $FH$ module affording the identity representation on $H$. Let  $\{ g_i\}$ denote representatives of cosets of  $H$ in $G$. The representation on this set is given by left multiplication on this set. Given some $g_i , g \in G$ we have 
\[g g_i \in g_j H \]
for some $g_j$ in the representative set. Thus, fixing some $a \in F$ we have
\begin{align*}
g( g_i \otimes a v)  &= g g_i \otimes a v\\
&= g_ j h \otimes a v\\
&= g \otimes a h_j v \\
&= g_j \otimes av 
\end{align*}
So, we see that $FG \otimes_{FH} V $ has the same action as does the permutation representation
\end{proof}
\section{Question 5}
\subsection{Question}
Fill in the details of the sketch of the construction of the character table of GL$_2(\mathbb{F}_p)$ given in lectures, as follows.
\begin{enumerate}
\item Compute the conjugacy classes in GL$_2(\mathbb{F})$ and their orders
\item Compute the characters of $I(\chi_1,\chi_2)$, and decompose it as a sum of irreducible characters.
\item Follow the approach in Lang to find the characters of some irreducible $(p-1)$-dimensional representations.
\item Show that you have computed the character table of GL$_2(\mathbb{F}_p)$.
\end{enumerate}
\subsection{Answer}
\begin{enumerate}
\item To compute the conjugacy classes in GL$_2(\mathbb{F})$ we first observe that if the matrix has eigenvalues in $\mathbb{F}_p$ then it is conjugate to a matrix of one of the following 3 forms (Jordan Canonical Form)
\[\left( \begin{array}{cc} a& 0\\0&b\end{array}\right)\quad \left( \begin{array}{cc} a& 0\\0&a\end{array}\right) \quad \left( \begin{array}{cc} a& 1\\0&b\end{array}\right).\]
One of these representatives is uniquely determined by the eigenvalues of a matrix.

We need to deal with the case where the matrix has no eigenvalues in $\mathbb{F}_p$. By Rational Canonical Form, any such matrix is conjugate to one of the form
\[\left( \begin{array}{cc} 0& -b\\1&-a\end{array}\right)\]
for $a, b$ coefficients of the characteristic polynomial.

Now that we have identified a choice of representative for every conjugacy class in the group we must count the number of distinct conjugacy classes of each form, and how many elements are in each of theses.

For the first form there are $\frac{1}{2}(p-1)(p-2)$ such classes, since the order of the eigenvalues does not matter, yet they must be distinct. To compute the number of elements in a class of this form, first observe that the stabilizer is the set of diagonal matrices. There are $(p-1)^2$ such matrices, so there are 
\[|G|/|Z(g)| = \frac{(p+1)p(p-1)^2}{(p-1)^2} = p(p+1) \]
elements in a fixed such conjugacy class, by the orbit-stabilizer theorem.

There are $p-1$ conjugacy classes of the one-eigenvalue form, as they are determined by some nonzero choice of eigenvalue. These commute with every element of the group, so the classes contain only one member each.

The non-diagonalizable, one eigenvalue form also has $p-1$ separate conjugacy classes, defined by the choice of (again nonzero) eigenvalue. These are stabilized by matrices of the form
\[\left( \begin{array}{cc} x& y \\ 0 & x \end{array} \right) \quad x \neq 0,\]
so again by the orbit-stabilizer theorem, one such conjugacy class contains
\[|G|/|Z(g)| = \frac{(p+1)p(p-1)^2}{(p-1)p} = (p+1)(p-1) \]
elements.

For the last type, there are $p(p-1)$ choices of eigenvalue, but the order doesn't matter so there are a total of $\frac{1}{2}p(p-1)$ choices of conjugacy class. By computing the number of elements of GL$_2(\mathbb{F}_p)$, subtracting the contribution from elements of the first three forms, then dividing, we find that each of these conjugacy classes has 
\[p(p-1)\]
elements.
\item The easy representations to see are the one dimensional ones. Just take any homomorphism $\varphi: \mathbb{F}_p^* \to \mathbb{C}^*$, and then compose it with the determinant. These representations are given by specifying which $p$th root of unity we take the generator of $\mathbb{F}_p$ to, so there are $p-1$ of them.

The characters for these representations are
\[\begin{array}{|c|c|c|c|c|}
\hline
& \mbox{Diagonalizable}&&  & \\
& \mbox{1 Eigenvalue}& \mbox{1 Eigenvalue}& \mbox{2 Eigenvalues}&\mbox{No Eigenvalues}\\
\hline
\varphi \circ det & \varphi(a)^2 & \varphi(a)^2 & \varphi(a b) & \varphi(b)\\
\end{array}
\]
\end{enumerate}

\section{Question 6}
\subsection{Question}
Define the \emph{kernel} of a character $\chi$ to be the set of $g \in G$ such that $\chi(1) =\chi(g)$. Show that the kernel of $\chi$ is a normal subgroup of $G$, and that every normal subgroup of $G$ is the intersection of the kernels of some set of irreducible characters of $G$.
\subsection{Answer}
First we show that the kernel is a normal subgroup.
\begin{proof}
Fixing some character $\chi$ we have 
\[ker\chi = \{ g \in G \mid \chi(1) = \chi(g) \}\]
so, assuming $x \in ker\chi$, $\chi(g^{-1} x g ) = \chi(x) = \chi(1)$, since characters are class functions. Therefore, $g^{-1} x g \in ker\chi$ and $ker\chi$ is normal, by definition.
\end{proof}

Now, we show that every normal subgroup of $G$ is the intersection of the kernels of a set of irreducible characters of $G$. 
\begin{proof}
First we will show that if $\chi$ is a character of $G$ which is a linear combination of some irreducible characters of $G$, 
 \[\chi = a_1 \chi_1 + a_2 \chi_2 + \cdots + a_n \chi_n\]
then $ker\chi= \bigcap\{ \ker\chi_i \}$

Observe that $\chi(g)  = \sum_i a_i \chi_i(g)  = \sum_i a_i \chi_i(1) = \chi(1)$ as $ \chi_i(g) \leq \chi_i(1) $. Therefore, $ker\chi \subseteq \bigcap \{ \ker\chi_i \}$. The reverse inclusion is obvious, and we get the desired  $ker\chi= \bigcap\{ \ker\chi_i \}$.

Now fix a normal subgroup $N \unlhd G$ and denoting by $A(G)$ the group algebra of $G$, define $\rho: G \to A(G/N)$ as
\[\rho_{gh}(f) (k) = f(kghN) = \rho_h (f) (kgN) =(\rho_g(\rho_h(f)))(k).\]
Clearly, this gives us a representation with $N \subseteq ker\rho$. 

Now define a function $\delta_N: G/N \to\mathbb{C}$ by
\[\delta_N(g) = \left\{ \begin{array}{ll} 1 & \mbox{if }g \in N \\ 0 & \mbox{otherwise} \end{array} \right. \]
Observe now that if $g \in \ker\rho$ we have $\rho_g(\delta_N) = \delta_N$, and $1 = \delta_N(N) = \delta_N(gN)$. Hence either $N = gN$, or $g \in N$. Furthermore,  as $g\in N, \ker \chi_\rho = N$. Therefore, $N = ker\chi = \bigcap ker \chi_i$, for $\chi_i$ irreducible, and we are done, by our earlier work.
\end{proof}

\section{Question 7}
\subsection{Question}
Using the character tables you have computed, determine all normal subgroups of $D_{2n}, Q_8, S_4, A_5,$ and $S_5$.
\subsection{Answer}
We employ the result of the previous exercise throughout. Since every normal subgroup is the intersection of the kernels of some set of irreducible characters, we can enumerate all of the normal subgroups by computing the kernels of the irreducible representations from the character tables, and looking at the intersections of these. 

\begin{enumerate}
\item First we have $D_{2n}$, $n$ odd.
\[\begin{array}{|c|c|c|cccc|c|}
\hline
&1&r&\cdot&\cdot&\cdot&r^{\frac{n-1}{2}}&s\\
\hline
\chi_1&1&1&\cdot&\cdot&\cdot&1&1\\
\hline
\chi_2&1&1&\cdot&\cdot&\cdot&1&-1\\
\hline
\chi_3&2&2 \cos{\left(\frac{2 \pi }{ n}\right)}&\cdot&\cdot&\cdot&2 \cos{((n-1) \pi / n)}&0\\
\hline
\cdot&\cdot&\cdot&\cdot&\cdot&\cdot&\cdot&\cdot\\
\cdot&\cdot&\cdot&\cdot&\cdot&\cdot&\cdot&\cdot\\
\chi_{(n+3)/2}&2&2\cos{(2(\frac{n+3}{12} -2)\pi/n)}&\cdot&\cdot&\cdot&\cdot&0\\
\hline
\end{array}\]

So the only proper, nontrivial normal subgroups is $\left<r\right>$.

Now the even case
\[\begin{array}{|c|c|c|c|cccc|c|c|}
\hline
&1&r&r^2&\cdot&\cdot&\cdot&r^{n/2}&s&rs\\
\hline
\chi_1&1&1&1&\cdot&\cdot&\cdot&1&1&1\\
\hline
\chi_2&1&-1&1&\cdot&\cdot&\cdot&\pm1&1&-1\\
\hline
\chi_3&1&1&1&\cdot&\cdot&\cdot&\pm1&1&-1\\
\hline
\chi_4&1&-1&1&\cdot&\cdot&\cdot&\pm1&1&-1\\
\hline
\chi_{5}&2&2\cos{(2\pi/n)}&\cdot&\cdot&\cdot&\cdot&2 \cos{(n \pi /n)}&&\\
\hline
\cdot&\cdot&\cdot&\cdot&\cdot&\cdot&\cdot&\cdot&\cdot&\cdot\\
\hline
\chi_{n/2 + 3} & 2 & 2 \cos{2(n/2-1) \pi / n}&\cdot&\cdot&\cdot&\cdot&\cdot&0&0\\
\hline
\end{array}\]
so the only proper, nontrivial normal subgroups are $\left<r\right>,\left<r^2, s\right>,\left<r^2, rs\right>$.

\item
The character table for $Q_8$ is
 \[\begin{array}{|c|c|c|c|c|c|}
\hline
&1&-1&i&j&k\\
\hline
\chi_1&1&1&1&1&1\\
\hline
\chi_2&1&1&-1&1&-1\\
\hline
\chi_3&1&1&1&-1&-1\\
\hline
\chi_4&1&1&-1&-1&1\\
\hline
\chi_5&2&-2&0&0&0\\
\hline
\end{array}\]
So all subgroups of $Q_8$ are normal.

\item 
The character table of $S_4$ is
\[
\begin{array}{|c|c|c|c|c|c|}
\hline
&1&6&8&6&3\\
&e&(12)&(123)&(1234)&(12)(34)\\
\hline
\chi_1&1&1&1&1&1\\
\hline
\chi_2&1&-1&1&-1&1\\
\hline
\chi_3&2&0&-1&0&2\\
\hline
\chi_4&3&1&0&-1&-1\\
\hline
\chi_5&3&-1&0&1&-1\\
\hline
\end{array}\]
So, the normal subgroups are $1, A_4, V, S_4$.

\item 
The character table of $A_5$ is
\[
\begin{array}{|c|c|c|c|c|c|}
\hline
&1&20&12&12&15\\
\hline
\chi_1&1&1&1&1&1\\
\hline
\chi_2&3&0& \frac{1+\sqrt5}{2}&\frac{1-\sqrt5}{2}&-1\\
\hline
\chi_3&4&1&-1&-1&0\\
\hline
\chi_4&3&0&\frac{1-\sqrt5}{2}&\frac{1+\sqrt5}{2}&-1\\
\hline
\chi_5&5&-1&0&0&1\\
\hline
\end{array}\]
So there are no nontrivial, proper normal subgroups.

\item 
The character table of $S_5$ is
\[
\begin{array}{|c|c|c|c|c|c|c|c|}
\hline
&1&10&20&30&24&15&20\\
&e&(12)&(123)&(1234)&(12345)&(12)(34)&(12)(345)\\
\hline
\chi_1&1&1&1&1&1&1&1\\
\hline
\chi_2&1&-1&1&-1&1&1&-1\\
\hline
\chi_3&4&2&1&0&-1&0&-1\\
\hline
\chi_4&4&-2&1&0&-1&0&1\\
\hline
\chi_5&5&-1&-1&1&0&1&-1\\
\hline
\chi_6&5&1&-1&-1&0&1&1\\
\hline
\chi_7&6&0&0&0&1&-2&0\\
\hline
\end{array}\]
\end{enumerate}
and we observe that the normal subgroups are just $S_5, A_5$ and the trivial one.


\end{document}
