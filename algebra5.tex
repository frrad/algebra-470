\documentclass[11pt]{article}



\usepackage[all]{xy}
\usepackage{fancyhdr}
\usepackage{amsthm}
\usepackage{amssymb}
\usepackage{setspace}
\pagestyle{fancyplain}

\begin{document}

\lhead{Frederick Robinson}
\rhead{Math 470: Algebra}



\title{Homework 5}
\author{Frederick Robinson}
\date{14 February 2011}
\maketitle




\section{Question 1}
\subsection{Question}
Make explicit the equivalences explained in lectures between $\{$maximal left ideals$\} \leftrightarrow \{$simple modules with a specified nonzero element$\}$ and $\{$simple modules$\} \leftrightarrow \{$maximal 2-sided ideals$\}$ in the case that $A = $ End$_kV$, $V$ a finite-dimensional $k$-vector space.
\subsection{Answer}
Maximal left ideals of End$_kV$ are endomorphisms whose kernels contain the same 1-dimensional subspace of $V$. That is, for any fixed 1 dimensional subspace of $V$ there is a corresponding maximal left ideal: all endomorphisms whose kernel contains that subspace.

\begin{proof}
The principle ideal generated by an endomorphism $A$ is the set of all endomorphisms whose kernels contain the kernel of $A$. Clearly any endomorphism whose kernel doesn't contain $A$ cannot be reached by composing $A$ with another endomorphism. Furthermore, so long as an endomorphism $B$ has kernel which contains the kernel of $A$, it is easy to construct another endomorphism which, when composed with $A$ give us $B$.

Now, we just observe that the ideals generated by endomorphisms with 1 dimensional kernels are maximal. If $A$ has 1-dimensional kernel, and $B$ is not in $(A)$, then either $B$ has no kernel, or the kernel of $A$ does not coincide with that of $B$. In the first case, $(B)$ is all of End$_k$, and in the second, $(A+B+C)$= End$_k V$ for some $C \in (A)$.

These are all of the left ideals since every principle ideal is contained in one of them.
\end{proof}

The only maximal two sided ideal is the trivial ideal.
\begin{proof}
The principle ideal generated by $A \in$ End$_k V$ contains all endomorphisms with the same dimension kernel. By operating one one side we can get all endomorphisms with the same kernel, and by operating on the other we can ``rotate" the kernel, while fixing its dimension.

Now just observe that, $A \neq 0$ we can add a finite number of such endomorphisms to get one which has no kernel. Since an endomorphism with no kernel generates all the endomorphisms $A= 0$.
\end{proof}

Simple modules of $V$ with specified choice of nonzero element are just the one dimensional subspace of $V$ spanned by that element.
\begin{proof}
We know from linear algebra that submodules of $V$ are defined by a set of vectors which span them. Clearly the submodules which do not have further (nontrivial) submodules are precisely those of dimension 1.
\end{proof}

Simple modules without choice of nonzero element are $k^n$. This is proven in a subsequent exercise (number 4). 

The correspondence is obvious for  $\{$simple modules$\} \leftrightarrow \{$maximal 2-sided ideals$\}$. For   $\{$maximal left ideals$\} \leftrightarrow \{$simple modules with a specified nonzero element$\}$ it is given by sending the subspace spanned by a vector $\vec{v}$ to the endomorphisms which annihilate it.

\section{Question 2}
\subsection{Question}
Show that if $Q$ denotes the quaternions, then $Q \otimes_\mathbb{R} \mathbb{C} \cong M_2 (\mathbb{C}).$
\subsection{Answer}
\begin{proof}

Recall that we can represent the quaternions as complex matrices by taking
\[a+bi+cj+dk \mapsto \left[ \begin{array}{cc} a+bi & c+di \\ - c+di & a-bi \end{array}\right]\]

Now observe that $\mathbb{Q} \otimes \mathbb{C} \cong \mathbb{C}[i,j,k]$ with the usual quaternion relations. This is a ring with the right dimension to be isomorphic to $M_2(\mathbb{C})$. Take a set of basis elements for $M_2(\mathbb{C})$
\[\left(\begin{array}{cc} 1 & 0 \\0 & 1\end{array} \right) ,\left(\begin{array}{cc} -i & 0 \\0 & -i\end{array} \right) ,\left(\begin{array}{cc} 0& 1 \\-1 & 0\end{array} \right) ,\left(\begin{array}{cc} 0 & i \\i & 0\end{array} \right) .\]
We just need to check that this is a basis. Linear combinations of these matrices are of the form
\[\left( \begin{array}{cc} a -b i & c + d i \\ -c+ d i & a -b i   \end{array} \right).\]
so the matrices span, as claimed.\end{proof}

\section{Question 3}
\subsection{Question}
Let $A$ be a ring and let $M$ be a semisimple left $A$-modules. Show that the following are equivalent.
\begin{enumerate}
\item $M$ is finitely generated.
\item $M$ is a direct sum of finitely many simple submodules.
\item $M$ is a sum of finitely many simple submodules.
\item $M$ is Noetherian.
\item $M$ is Artinian.
\end{enumerate}
\subsection{Answer}
First we'll show $1 \Rightarrow 2$.
\begin{proof}
Suppose $M$ is finitely generated, with generators $M_1, \dots, M_m$. Since $M$ is semisimple it is the direct sum of simple submodules
\[M = \bigoplus_{n \in I} S_n.\]
Now fix some $M_i$. $M_i \in M$ so it can be expressed as part of the direct sum. However, it only takes nonzero contribution from finitely  many of the $S_n$ by definition of the direct sum. Thus, repeating for all $M_i$ we can isolate finitely many of the $S_n$ whose direct sum includes all of the $M_i$ and therefore all of the module.\end{proof}

$2 \Rightarrow 3$ is trivially true, and we proved $3 \Rightarrow 2$ during class.

Now, $3 \Rightarrow 4$.
\begin{proof}
Let $M_1 \subset M_2 \subset \dots$ be an ascending chain. However, since the space decomposes as a finite direct sum of simple submodules each $M_i$ is the direct sum of a finite number of these submodules, and so is at the longest
$S_1 \subset S_1 \oplus S_2 \subset \dots \subset M$.
\end{proof}

The proof that $3 \Rightarrow 5$ follows similarly.

Now we'll prove that $4 \Rightarrow 1$
\begin{proof}
Let $Am$ be a principle submodule for some $m \in M$. Then $M = A m \oplus M'$ for $M'$ a finitely generated submodule of $M$. Clearly $Am$ is also finitely generated.
\end{proof}


Finally, we finish by showing that $5 \Rightarrow 2$
\begin{proof}
Suppose towards a contradiction that $M$ is Artinian but not a finite direct sum of simple submodules. Consider $M = \bigoplus_1^\infty S_i$ with the descending chain $N_{k+1} \supseteq N_k$ where $N_k = \bigoplus_{i=k}^\infty M_i$. This does not satisfy the descending chain condition,  a contradiction.
\end{proof}


\section{Question 4}
\subsection{Question}
Prove that if $D$ is a division algebra $M_n(D)$ is a simple ring.
\subsection{Answer}
\begin{proof}
First we'll show that $M_n (D) $ is a two-sided ideal.  Fix some nonzero $a \in M_n(D)$. Clearly there is some nonzero entry in $a$, say the $i,j$th. If we denote by $\vec{v}_ij(d) $ the matrix which has $d$ in the $i,j$th place and 0s everywhere else we can form the product
\[\vec{v}_{ix}(b^{-1}) a \vec{v}_{yj}(d) = \vec{v}_{ij}(d).\]
Now that we've constructed an arbitrary elementary matrix as a product of $a$ we know that $M_n(D)$ is a two-sided ideal, as claimed.

By the equivalence of the first exercise we know that $M_n(D)$ is simple module, simple ring.
\end{proof}

\section{Question 5}
\subsection{Question}
Explicitly write $\mathbb{C}[\mathbb{Z}/n \mathbb{Z}]$ as a product of simple rings.
\subsection{Answer}
Denote by $\zeta_n$ an $n$th root of unity. Then, by the Chinese Remainder Theorem
\[\mathbb{C}[\mathbb{Z}/n\mathbb{Z}] = \bigoplus_{k=0}^{n-1} \mathbb{C} \left[\frac{1}{n} \sum_{i=0}^{n-1} \zeta_n^{k_i} a^i\right].\]
However since $a^n=1$ we have
\[ \left(\frac{1}{n} \sum_{i=0}^{n-1} \zeta_n^{k_i} a^i\right)^2 = \frac{1}{n} \sum_{i=0}^{n-1} \zeta_n^{k_i} a^i .\]
Hence, $\mathbb{C} \left[\frac{1}{n} \sum_{i=0}^{n-1} \zeta_n^{k_i} a^i\right]$ is  a ring of dimension one over $\mathbb{C}$ which is simple since $\frac{1}{n} \sum_{i=0}^{n-1} \zeta_n^{k_i} a^i$ is an identity. 
\[\mathbb{C}[\mathbb{Z}/n\mathbb{Z}] = C_1 \times \cdots C_n. \]

\section{Question 6}
\subsection{Question}
Explicitly write $\mathbb{C}[S_3]$ as a product of simple rings.
\subsection{Answer}
\[\mathbb{C}[S_3] = \mathbb{C}[\frac{1}{6} ( [Id]+ [(12)] + [(23)]  +[(13)] + [(123)] + [(132)])] \]
\[ \oplus \mathbb{C}[\frac{1}{6} ( [Id]- [(12)] - [(23)]  -[(13)] + [(123)] + [(132)])] \]
\[ \oplus \mathbb{C}[\frac{1}{6} ( [Id]- [(12)] - [(23)]  -[(13)] + [(123)] + [(132)])] \]
\[ \oplus \mathbb{C}[  [Id]- \omega [(123)]  + \omega^2 [(132)]  ,\]\[ [Id] + \omega [(132)] + \omega^2 [(123)]     , \]\[[(12)] + \omega [(123)]  + \omega^2 [(13)]  ,\]\[ [(12)]- \omega [(13)]  + \omega^2 [(23)]  \]

Equality holds, since we just write a new basis for the original space. It's also a direct sum, so long as we're dealing with rings. It there fore suffices to show that the third term of our direct sum is a simple ring. 

We show that is simple from the character table of $S_3$. $\mathbb{C}[S_3] \cong  \mathbb{C} \times \mathbb{C} \times M_2(\mathbb{C})$, so we see that it's a ring with dimension 4 over $\mathbb{C}$. Since it cannot be reducible, we have simplicity as desired.

We can check that it is closed with a straightforward computation.

\section{Question 7}
\subsection{Question}
Explicitly write $\mathbb{Q}[Q_8]$ as a product of simple rings ($Q_8$ is the quaternion group of order 8).
\subsection{Answer}
We can write $Q_8 = \left< a,b \mid a^4 =b^4 =1 , a^2 = b^2, a^{-1}ba = b^{-1} \right>$, where $a=i, b=j$. Now, define simple rings by
\[A_1=\mathbb{Q} \left[\frac{1}{8} \left(1+a+a^2+a^3+b + ab+a^2 b +a^3 b\right)\right]\]
\[A_2=\mathbb{Q} \left[\frac{1}{8} \left(1+a+a^2+a^3-b - ab-a^2 b -a^3 b\right)\right]\]
\[A_3=\mathbb{Q} \left[\frac{1}{8} \left(1-a+a^2-a^3+b - ab+a^2 b -a^3 b\right)\right]\]
\[A_4=\mathbb{Q} \left[\frac{1}{8} \left(1-a+a^2-a^3-b + ab-a^2 b +a^3 b\right)\right]\]
The argument above gives simplicity. Finally define 
\[A_5 = \mbox{span}\{1-a^2, a-a^3,b-a^2b,ab-a^3b \}\] and let the inverse images be 
\[\Delta_1 = \frac{1}{2}(a-a^2), \Delta_i(a-a^2), \Delta_j=(b-a^2b), \Delta_k=(ab-a^3b)\]
By the relations on the Quaternions, $\Delta_1$ is the identity and $\Delta_5$ is closed and therefore $A_0$ is a simple subring. Hence we have $A = A_1 \times A_2 \times A_3 \times A_4 \times A_5$ which is enough by Artin  Wedderberg.
\section{Question 8}
\subsection{Question}
Show that the dual representation $V^*$ really is a representation, and compute its character in terms of the character of $V$.
\subsection{Answer}
The dual representation is
\[g f(v) = f(g^{-1}(v)).\]
Then, for $g_1, g_2$  we get
\[(g_1g_2)f(v) = f((g_1g_2)^{-1}(v)) = f(g_2^{-1}g_1^{-1}(v))\]
\[=g_2(f(g^{-1}_1(v)) = g_1(g_2(f(v)).\]
So, it's a group homomorphism as desired and the dual is a representation, as claimed. 

The character of $B$ is the trace of the representation, or the sum of the eigenvalues. So, let $A$ be a representation. I$A$ is of finite order so $A^m = Id$ and its eigenvalues are roots of unity. Therefore for eigenvalues $\lambda_i$ of $A$, $\lambda_i^{-1} = \overline{\lambda_i}$, and the character of the dual is merely
\[Tr(A^{-1}) = \sum \lambda_i^{-1} = \sum \overline{\lambda_i} = \overline{Tr(A)}\]
\[\Rightarrow \chi_{V^*}(g) = \overline{\chi_v(g)}\]

\section{Question 9}
\subsection{Question}
Compute the character tables of $D_{2n}$ and $Q_8$. What do you notice about the character tables of $D_8$ and $Q_8$?
\subsection{Answer}
We can represent $Q_8 = \left< i,j \mid i^4, i^2 = j^2, i^{-1}ji = j^{-1}\right>$ . The conjugacy classes are $\{1\}, \{-1\}, \{i, -i\},\{j,-j\}, \{k,-k\}$. Since $Q_8$ mod the commutator is $\mathbb{Z}/2\mathbb{Z} \times \mathbb{Z} / 2 \mathbb{Z}$ four of the characters must have degree 1. The last character has degree given by $4+ n^2 = 8 \Rightarrow n = 2$.

We can fill in the last row by orthogonality to get
\[\begin{array}{|c|c|c|c|c|c|}
\hline
&1&-1&i&j&k\\
\hline
\chi_1&1&1&1&1&1\\
\hline
\chi_2&1&1&-1&1&-1\\
\hline
\chi_3&1&1&1&-1&-1\\
\hline
\chi_4&1&1&-1&-1&1\\
\hline
\chi_5&0&0&0&0&0\\
\hline
\end{array}\]

Now we'll compute the character table for $D_n$. We have a presentation as 
\[D_{2n} = \left< r,s \mid r^n = s^2 = (rs)^2 = 1 \right>.\] So, we may infer that the abelianization is 
\[D_{2n}^{ab} = \left< r,s \mid rs = sr , s^2 = r^{GCD(2n)} = 1 \right>.\]
Hence if  $n$ is odd we have $\mathbb{Z}/2\mathbb{Z}$. If $n$ is even we get $\mathbb{Z}\2\mathbb{Z} \times \mathbb{Z}/2\mathbb{Z}$. 

The one dimensional representations are either of degree 2 or degree 4 depending on if $n$ is odd or even.

We'll do the case of $n$ odd first

There are 2 1-dimensional representations. The conjugacy classes are given by $\{1\},\{r\},\{r^2\}, \dots, \{r^{(n-1)/2}\},\{s\}$. There are two representations mapping $r \mapsto 1, s \mapsto \pm 1 $ and $(n-1)/2$ representations which map $r$ to rotation matrices and
\[s \mapsto \left( \begin{array}{cc} 0&1 \\ 1& 0 \end{array} \right)\]

The character for this representation is $2 \cos{(2 k \pi /n)}$, corresponding to the rotation matrix and 0 corresponding to matrix we just gave. For $n$ the table is
\[\begin{array}{|c|c|c|cccc|c|}
\hline
&1&r&\cdot&\cdot&\cdot&r^{\frac{n-1}{2}}&s\\
\hline
\chi_1&1&1&\cdot&\cdot&\cdot&1&1\\
\hline
\chi_2&1&1&\cdot&\cdot&\cdot&1&-1\\
\hline
\chi_3&2&2 \cos{\left(\frac{2 \pi }{ n}\right)}&\cdot&\cdot&\cdot&2 \cos{((n-1) \pi / n)}&0\\
\hline
\cdot&\cdot&\cdot&\cdot&\cdot&\cdot&\cdot&\cdot\\
\cdot&\cdot&\cdot&\cdot&\cdot&\cdot&\cdot&\cdot\\
\chi_{(n+3)/2}&2&2\cos{(2(\frac{n+3}{12} -2)\pi/n)}&\cdot&\cdot&\cdot&\cdot&0\\
\hline
\end{array}\]

Now the even case

Then, we have from above that there are 4 characters of degree 1 which map $r \mapsto \pm 1, s \mapsto \pm 1$, and for the other representations (of which there are $n/2-1$, $r$ maps to the rotation matrices, and $s$ to the reflection, each with characters $2 \cos{(2 k \pi /n)}$ and 0 as before.

\[\begin{array}{|c|c|c|c|cccc|c|c|}
\hline
&1&r&r^2&\cdot&\cdot&\cdot&r^{n/2}&s&rs\\
\hline
\chi_1&1&1&1&\cdot&\cdot&\cdot&1&1&1\\
\hline
\chi_2&1&-1&1&\cdot&\cdot&\cdot&\pm1&1&-1\\
\hline
\chi_3&1&1&1&\cdot&\cdot&\cdot&\pm1&1&-1\\
\hline
\chi_4&1&-1&1&\cdot&\cdot&\cdot&\pm1&1&-1\\
\hline
\chi_{5}&2&2\cos{(2\pi/n)}&\cdot&\cdot&\cdot&\cdot&2 \cos{(n \pi /n)}&&\\
\hline
\cdot&\cdot&\cdot&\cdot&\cdot&\cdot&\cdot&\cdot&\cdot&\cdot\\
\hline
\chi_{n/2 + 3} & 2 & 2 \cos{2(n/2-1) \pi / n}&\cdot&\cdot&\cdot&\cdot&\cdot&0&0\\
\hline
\end{array}\]

In the special case of $n=4$ this character table agrees with that of $Q_8$.

\section{Question 10}
\subsection{Question}
Show that if $V$ is a representation of a finite group $G$, then $V\otimes V$ is a direct sum of two subrepresentations Sym$^2V$ and $\wedge^2 V$ where Sym$^2V$ is the subspace generated by the tensors $v \otimes w + w \otimes v$, and you should describe $\wedge^2V$ in a similar fashion. Compute the characters of these subrepresentations in terms of the character of $V$.
\subsection{Answer}
Given the setup we have $\chi_{\rho \otimes \rho}(g) = \chi_{sym^2}(g) + \chi_{\wedge^2}(g)$. Thus, it suffices to compute the character of one.

Then, we can choose a basis set, and element of the group in such a way that $g \in G, \rho(g)$ is upper triangular. Now, if our basis of $V$ is given by $v_i$ then $v_i \otimes v_j + v_j \otimes v_i$ for $i \leq j$ is a basis for Sym$^2 V$.

Observe that
\[ \chi_{sym^2}(g) = Tr(\rho \otimes \rho (g)|_{sym^2 V})\]
and
\[ \rho \otimes \rho (v_i \otimes v_j + v_j \otimes v_i) = \sum_{k=1}^n \sum_{l=1}^n a_{ik} a_{jl} (v_i \otimes v_j + v_j \otimes v_i).\]

Now to get the character we have
\[\chi_{sym^2}(g) = \sum_{1 \leq i < j \leq n} (a_{ij}a_{ji}+a_{ii}a_{jj}) = \sum_{1 \leq i < j \leq n} a_{ii}a_{jj} + \sum_{i = 1} ^n a_{ii}^2\]
by upper triangularity. Then,
\[ = \frac{1}{2} ( \sum_{i=1}^n  a_{ii} ) ^2 + \frac{1}{2} \left( \sum_{i=1}^n a_{ii}^2 \right) = \frac{1}{2} ( \chi_\rho (g))^2 + \frac{1}{2} (\chi_\rho (g^2)) \]
and finally we conclude
\[\chi_{\wedge^2(g)} = \frac{1}{2} ( \chi_\rho(g))^2 - \frac{1}{2} (\chi_\rho(g^2)).\]
\section{Question 11}
\subsection{Question}
Compute the character table of $S_5$ in a similar way to the manner in which we computed the character table of $S_4$. [Hint: you may find the previous exercise helpful.]
\subsection{Answer}
\[
\begin{array}{|c|c|c|c|c|c|c|c|}
\hline
&1&10&20&30&24&15&20\\
&e&(12)&(123)&(1234)&(12345)&(12)(34)&(12)(345)\\
\hline
\chi_1&1&1&1&1&1&1&1\\
\hline
\chi_2&1&-1&1&-1&1&1&-1\\
\hline
\chi_3&4&2&1&0&-1&0&-1\\
\hline
\chi_4&4&-2&1&0&-1&0&1\\
\hline
\chi_5&5&1&-1&-1&0&1&1\\
\hline
\chi_6&5&-1&-1&1&0&1&-1\\
\hline
\chi_7&6&0&0&0&1&-2&0\\
\hline
\end{array}\]
\begin{proof}
There are seven representations given by the form of their cycle decomposition we can compute $|\Gamma_1| =1 ,|\Gamma_2| =10 ,|\Gamma_3| =20 ,|\Gamma_4| =30 ,|\Gamma_5| =24 ,|\Gamma_6| =15 ,|\Gamma_7| =20 .$ We get the trivial character and the sign character ($\chi_1, and \chi_2$) right away. Also, we can include (from class) the character given by one less than the number of fixed points ($\chi_3$).


There may only be two characters of degree 1. The characters come from maps of the form $S_5 / N \to C_m, N \lhd S_5$, but $A_5$ has index 2. Since it is the commutator subgroup, there are only 2 degree 1 characters.

Since the product of two characters is again a character, we get $\chi_4=\chi_2\chi_3$ for free. This comprises the first 4 rows of the table.

Now, recall that the sum of the squares of the degrees is the degree of $S_5$. In particular then, the remaining three degrees have squares which sum to 86. Since none of these degrees may be one, the only possibility is $5^2+5^2+6^2=86.$

We'll say the degree of $\chi_5,\chi_6$ is 5 and the degree of $\chi_7$ is 6. Since the product $\chi_6 \chi_2$ is an irreducible character of degree 5 we have $\chi_5 = \chi_6 \chi_2$. Similarly,  $\chi_2 \chi_7 = \chi_1 \chi_7$.

Now we know
\[
\begin{array}{|c|c|c|c|c|c|c|c|}
\hline
&1&10&20&30&24&15&20\\
&e&(12)&(123)&(1234)&(12345)&(12)(34)&(12)(345)\\
\hline
\chi_1&1&1&1&1&1&1&1\\
\hline
\chi_2&1&-1&1&-1&1&1&-1\\
\hline
\chi_3&4&2&1&0&-1&0&-1\\
\hline
\chi_4&4&-2&1&0&-1&0&1\\
\hline
\chi_5&5&a&&b&&&c\\
\hline
\chi_6&5&-a&&-b&&&-c\\
\hline
\chi_7&6&0&&0&&&0\\
\hline
\end{array}\]

Finally we will finish by repeated application of orthogonality rules. Both the row orthogonality from class, and the second orthogonality rule (see for instance Dummit and Foote p872) which states that the inner product of two rows is 0 for different rows and the size of the group divided by the size of the conjugacy class  for the same row.

Using the second rule on the 2nd and 4th column gives us $2+ab+ab=0 \Rightarrow ab = -1$. So $a$ is either 1 or -1 and $b = -a$.

\[
\begin{array}{|c|c|c|c|c|c|c|c|}
\hline
&1&10&20&30&24&15&20\\
&e&(12)&(123)&(1234)&(12345)&(12)(34)&(12)(345)\\
\hline
\chi_1&1&1&1&1&1&1&1\\
\hline
\chi_2&1&-1&1&-1&1&1&-1\\
\hline
\chi_3&4&2&1&0&-1&0&-1\\
\hline
\chi_4&4&-2&1&0&-1&0&1\\
\hline
\chi_5&5&a&&-a&&&c\\
\hline
\chi_6&5&-a&&a&&&-c\\
\hline
\chi_7&6&0&&0&&&0\\
\hline
\end{array}\]
The same trick with the 4th and 7th columns tells us that $2-2ac=0 \Rightarrow ac=1 \Rightarrow c = a$.

\[
\begin{array}{|c|c|c|c|c|c|c|c|}
\hline
&1&10&20&30&24&15&20\\
&e&(12)&(123)&(1234)&(12345)&(12)(34)&(12)(345)\\
\hline
\chi_1&1&1&1&1&1&1&1\\
\hline
\chi_2&1&-1&1&-1&1&1&-1\\
\hline
\chi_3&4&2&1&0&-1&0&-1\\
\hline
\chi_4&4&-2&1&0&-1&0&1\\
\hline
\chi_5&5&a&&-a&&&a\\
\hline
\chi_6&5&-a&&a&&&-a\\
\hline
\chi_7&6&0&&0&&&0\\
\hline
\end{array}\]

We choose $a=1$ since we know it's either $1$ or negative one, and choosing one of these amounts to choosing a label for $\chi_5$ or $\chi_6$. Now, by looking at rows 3 and 4 together, 4 and 5 together, and 4 and 6 together we can glean a bit more information. Finally use the column orthogonality property on the columns with unknowns with themselves.


\[
\begin{array}{|c|c|c|c|c|c|c|c|}
\hline
&1&10&20&30&24&15&20\\
&e&(12)&(123)&(1234)&(12345)&(12)(34)&(12)(345)\\
\hline
\chi_1&1&1&1&1&1&1&1\\
\hline
\chi_2&1&-1&1&-1&1&1&-1\\
\hline
\chi_3&4&2&1&0&-1&0&-1\\
\hline
\chi_4&4&-2&1&0&-1&0&1\\
\hline
\chi_5&5&1&a&-1&b&c&1\\
\hline
\chi_6&5&-1&a&1&b&c&-1\\
\hline
\chi_7&6&0&\sqrt{2}\sqrt{1-a^2}&0&\sqrt{1-2b^2}&\sqrt{6-2c^2}&0\\
\hline
\end{array}\]
Finally, since we need to have integer values in all of these places we narrow down our choices to $a= \pm 1, c = \pm 1, b = 0$. Using the column orthogonality with the first column, we can pin down that $a$ and $b$ are negative and positive respectively.
\end{proof}

\end{document}
